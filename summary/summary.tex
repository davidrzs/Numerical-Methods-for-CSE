\documentclass{sciposter}
\usepackage[dvipsnames,usenames,svgnames,table,x11names, rgb, html]{xcolor} 
\usepackage{lipsum}
\usepackage{epsfig}
\usepackage{amsmath}
\usepackage{amssymb}
\usepackage[german]{babel}
\usepackage{geometry}
\usepackage{multicol}
\usepackage{graphicx}
\usepackage{tikz}
\usepackage{wrapfig}
\usepackage{gensymb}
\usepackage[utf8]{inputenc}
\usepackage{empheq}


\geometry{
 landscape,
 a1paper,
 left=5mm,
 right=50mm,
 top=5mm,
 bottom=50mm,
 }


\usepackage{todonotes}

\newcommand{\TODO}[1]{\todo[inline, color=red!40]{#1}}

%BEGIN LISTINGDEF



\usepackage{listings}
\usepackage{inconsolata}




\definecolor{background}{HTML}{FAFAFA}
\definecolor{comment}{HTML}{ABB0B6}
\definecolor{keywords}{HTML}{55B4D4}
\definecolor{basicStyle}{HTML}{6C7680}
\definecolor{variable}{HTML}{001080}
\definecolor{string}{HTML}{86B300}


\lstset{
	% How/what to match
	sensitive=true,
	% Border (above and below)
	frame=single,
	% Extra margin on line (align with paragraph)
	xleftmargin=\parindent,
	% Put extra space under caption
	belowcaptionskip=1\baselineskip,
	% Colors
	backgroundcolor=\color{background},
	basicstyle=\color{basicStyle}\ttfamily,
	keywordstyle=\color{keywords},
	commentstyle=\color{comment},
	stringstyle=\color{string},
	numberstyle=\color{sviolet},
	identifierstyle=\color{variable},
	% Break long lines into multiple lines?
	breaklines=true,
	% Show a character for spaces?
	showstringspaces=false,
	tabsize=2
}

%END LISTINGDEF


\newcommand*\widefbox[1]{\fbox{\hspace{2em}#1\hspace{2em}}}


\newlength\dlf  % Define a new measure, dlf
\newcommand\alignedbox[2]{
% Argument #1 = before & if there were no box (lhs)
% Argument #2 = after & if there were no box (rhs)
&  % Alignment sign of the line
{
\settowidth\dlf{$\displaystyle #1$}
    % The width of \dlf is the width of the lhs, with a displaystyle font
\addtolength\dlf{\fboxsep+\fboxrule}
    % Add to it the distance to the box, and the width of the line of the box
\hspace{-\dlf}
    % Move everything dlf units to the left, so that & #1 #2 is aligned under #1 & #2
\boxed{#1 #2}
    % Put a box around lhs and rhs
}
}
\usepackage{graphicx,url}

%BEGIN TITLE
\title{\huge{Numerical Methods for CSE}}

\author{\large{David Zollikofer}}
%END TITLE


\newcommand{\psection}[1]{\par \textbf{\large#1}}

\begin{document}
%\fontfamily{phv}\selectfont

\maketitle



\begin{multicols}{2}
\section{Eigen \& C++}

\psection{Constructing Matrices and Vectors}
\begin{lstlisting}[language=C++]
// same data different shape
Map<Matrix<TYPE, Dynamic , Dynamic>> newMat(M.data(),newRows,newCols);
// (Deep) copy data of M
MatrixXd xMat = MatrixXd::Map(M.data(), n, n);
\end{lstlisting}


\psection{Sizes, Rows, Cols, Resize}
\begin{lstlisting}[language=C++]
A.size() // total number of elements in matrix
A.rows() // number of rows
A.cols() // number of cols
A.resize(10,100) // resized to 10 rows and 100 columns
\end{lstlisting}


\psection{Filling matrices}
\begin{lstlisting}[language=C++]
A << R,	v // can also be done using blocks. 
u .transpose( ) ,  0;
B << A, A, A;     // B is three horizontally stacked A's.
A.fill(10);       // Fill A with all 10's.

MatrixXd::Identity(rows,cols)          
C.setIdentity(rows,cols)                 
MatrixXd::Zero(rows,cols)                
C.setZero(rows,cols)                      
MatrixXd::Ones(rows,cols)          
C.setOnes(rows,cols)                 
MatrixXd::Random(rows,cols)  // uniform random numbers in (-1,1).
C.setRandom(rows,cols)                 
VectorXd::LinSpaced(size,low,high)        
v.setLinSpaced(size,low,high)         
VectorXi::LinSpaced(((hi-low)/step)+1, low,low+step*(size-1))
\end{lstlisting}



\psection{Reshaping using Map}
\begin{lstlisting}[language=C++]
// same data different shape
Map<Matrix<TYPE, Dynamic , Dynamic>> newMat(M.data(),newRows,newCols);
// (Deep) copy data of M
MatrixXd xMat = MatrixXd::Map(M.data(), n, n);
\end{lstlisting}


\psection{Solving Equations}
\begin{lstlisting}[language=C++]
// Solve Ax = b. Result stored in x
x = A.ldlt().solve(b));  // A sym. p.s.d.    #include <Eigen/Cholesky>
x = A.llt() .solve(b));  // A sym. p.d.      #include <Eigen/Cholesky>
x = A.lu()  .solve(b));  // Stable and fast. #include <Eigen/LU>
x = A.qr()  .solve(b));  // No pivoting.     #include <Eigen/QR>
x = A.svd() .solve(b));  // Stable, slowest. #include <Eigen/SVD>
// .ldlt() -> .matrixL() and .matrixD()
// .llt()  -> .matrixL()
// .lu()   -> .matrixL() and .matrixU()
// .qr()   -> .matrixQ() and .matrixR()
// .svd()  -> .matrixU(), .singularValues(), and .matrixV()
\end{lstlisting}

\psection{Reductions}
\begin{lstlisting}[language=C++]
R.minCoeff()              
R.maxCoeff()              
s = R.minCoeff(&r, &c)   
R.sum()                  
R.colwise().sum()      
R.prod()                
R.rowwise().prod()        
R.trace()                
(a > 2).all() // if all a_i > 2 true             
R.rowwise().all()     
(a > 2).any()    // if any a_i > 2 true        
(a > 2).count() // count a_i > 2 
R.colwise().any()  
x.norm()                 
x.squaredNorm()       
x.dot(y)                 
x.cross(y)     //Requires #include <Eigen/Geometry>
\end{lstlisting}


\TODO{triangulare view etc, diagonal etc .}

\TODO{sparse formats}
\TODO{CRS / COO format}
\TODO{triplet format and how to handle it}

\section{Direct Methods for (Square) Linear Systems of Equations}

\section{Direct Methods for Linear Least Squares}

\section{Filtering Algorithms}

\section{Data Interpolation and Data Fitting in 1D}

\section{Approximating Functions in 1D}

\section{Numerical Quadrature}

\section{Iterative Methods for Non-Linear Systems of Equations}

\section{Computation of Eigenvalues and Eigenvectors}

\section{Krylov Methods for Linear Systems of Equations}

\section{Numerical Integration - Single Step Methods}

\section{Single Methods for Stiff Initial Value Problems}


	

\section{Lambda Functions}
\begin{lstlisting}[language=C++]
  auto phi1 = [](int t){return 1/t;};
  auto phi2 = [](int t){return 1/(t*t);};
  auto phi3 = [](int t){return 1/t;};
  auto phi4 = [](int t){return 1/t;};
\end{lstlisting}





%Example Listing
\begin{lstlisting}[language=c++]
  auto phi1 = [](double t){return 1/t;};

b.unaryExpr(phi1);
\end{lstlisting}


\end{multicols}
\end{document}